\graphicspath{{introduction/fig/}}

\chapter{Introduction}
\label{chap:introduction}

In the field of superconducting electronics the presence of excess low frequency flux noise has puzzled many. When a phenomenon is understood it can be modelled. When it can be modelled it can be designed for. The importance of designing for low frequency flux noise cannot be understated. In applications such as biomagnetism and quantum computing the low frequency flux noise has been identified as a limiting factor in the performance of the relevant superconducting electronics \cite{KochModel}. \par
This project will focus on superconducting quantum interference device (SQUID) sensors. In these devices there are a couple of sources of noise. These sources include thermal white noise from shunt resistors, 1/f noise from thermally activated critical current variations in Josephson Junctions and 1/f flux noise. The origin of the 1/f flux noise is an open question. \par
If one works back from the spectrum of the flux noise to try and determine the properties of the stochastic process that could produce the same spectrum, one would find that a sum of independently distributed random telegraph signal (RTS) processes could produce the desired spectrum \cite{fluxNoiseSquidsStevenAnton}. Although it is widely accepted that this process correctly models the flux noise, there is no consensus on the exact parameters of the RTS process \cite{fluxNoiseSquidsStevenAnton}. \par
This project will be largely based on the work done by S.M. Anton et al. \cite{fluxNoiseSquidsStevenAnton}. As such, I will accept that the flux noise is a consequence of surface defects on the surface of the superconducting structure. \par
In \cite{fluxNoiseSquidsStevenAnton}, the authors developed a numerical framework for calculating the mean square flux noise (MSFN) figure due to surface defects on SQUID washers. The original paper developed the framework and applied it to a very limited geometric structure. The study used an old version of InductEx. The goal of this project is to implement this framework for an arbitrary structure. It will also aim to use the vastly improved upon InductEx software package and TetraHenry engine. Most notably, the framework will be adapted to work with triangular and tetrahedral meshing. \par
This report will start with a review of the theory and applications of superconductors. I will then review the previous attempts at modelling flux noise. Next I will report on the design process. I will end with a description of my testing methodology as well as the results and conclusions of my testing. 
