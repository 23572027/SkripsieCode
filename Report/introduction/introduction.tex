\graphicspath{{introduction/fig/}}

\chapter{Introduction}
\label{chap:introduction}


\section{Theory and applications of superconductivity}
In order to understand and apply the methods described in \cite{fluxNoiseSquidsStevenAnton} one must first understand the basic theory behind superconductivity as well as some examples of how this theory is used in practice.

\subsubsection{Superconductivity}
\textcolor{red}{REVIEW NEEDED}
Since the first discovery of superconductivity a couple of successfully theories have been put forth to explain the phenomenon. The London theory is a framework that describes the qualitative behaviour of superconductors and correctly describes perfect diamagnetism and zero resistance but fails to explain the effect on a microscopic level \cite{Golubov_1998}. The London equations (eq. \ref{london1} and eq. \ref{london2}) \cite{Tinkham_2015} is an addition to Maxwell's equations.
\begin{equation}
    E = \frac{\partial}{\partial t}(\Lambda J_s)
    \label{london1}
\end{equation}
\begin{equation}
    h = -c \nabla\times (\Lambda J_s)
    \label{london2}
\end{equation}
Here $\Lambda$ is a phenomenological parameter determined through experimentation. The London equations allow us to calculate the current distribution in a superconductor which is very important to the goal of this project. BCS theory put forth a microscopic model of superconductors and explains the phenomenon as a quantum mechanical effect. The details are out of scope for this project but on a crude qualitative level BCS theory can be explained by the pairing of electrons in the crystal lattice of the superconductor allowing them to be considered one particle. These particles are known as cooper-pairs. At extremely low temperatures the formation of these cooper pairs are energetically favorable \cite{Feynman_Leighton_Sands_2013}. Electrons pairs in this state can flow through the superconductor unimpeded. BCS theory is the most successful model of superconductivity discovered to date. 
\subsubsection{The Josephson junction}
In superconducting electronics the active component is the Josephson junction \cite{Duzer_1999_Princip_Super}. The Josephson junction refers to a situation where two superconductors are connected through a thin non-cunduction barrier. If this barrier is thin enough one can observe what is known as the Josephson effect. This phenomena is can be explained by considering the effect of quantum tunneling of cooper pairs through the non-conductive boundary. For a sufficiently large barrier one can express the ensemble average wave function in each superconductor independently\cite{Duzer_1999_Princip_Super}:
\begin{equation}
    \Psi = |\Psi(\Vec{r})| \exp{\{i\theta(\Vec{r})\}}
    \label{eq:ensembleWave}
\end{equation}
This is due to the fact that it is energetically favorable for cooper pairs in close proximity to one another to lock phases \cite{Duzer_1999_Princip_Super} allowing one to express a large collection of these cooper pairs as one ensemble wave function. The idea that it is energetically favorable for cooper-pairs in close proximity to one another extends to the situation where the superconductors are separated by an insulating boundary. When the barrier is sufficiently small the energy of the system can be reduced by the coupling of wave functions in their respective superconductors \cite{Duzer_1999_Princip_Super}. This results in cooper-pairs being able to move across the boundary without energy loss. Following the derivation in \cite{Feynman_Leighton_Sands_2013} one can describe the system behavior using Schrodinger's equation when a voltage is applied to the junction \cite{Feynman_Leighton_Sands_2013}: 
\begin{equation}
    i\hbar \frac{\partial\Psi_1}{\partial t} = U_1\Psi_1 +K\Psi_2 
    \label{eq:shrod1}
\end{equation}
\begin{equation}
    i\hbar \frac{\partial\Psi_2}{\partial t} = U_2\Psi_2 +K\Psi_1
    \label{eq:shrod2}
\end{equation}
Here $U_1 = qV/2$ and $U_2 = -qV/2$ \cite{Feynman_Leighton_Sands_2013} refer to the energy of the two wave functions and $K$ refers to the coupling energy between each wave function. By taking \ref{eq:ensembleWave} and setting $|\Psi(\Vec{r})|$ to $\sqrt{\rho}$ where $\rho$ refers to the cooper pair density in the super conductor one can substitute the result into eq. \ref{eq:shrod1} and \ref{eq:shrod2}. One can obtain the result that the current through the junction is described by equation \ref{eq:joseph} \cite{Feynman_Leighton_Sands_2013}.
\begin{equation}
    J = J_0sin(\theta_2 - \theta_1)
    \label{eq:joseph}
\end{equation}
The phases of the currents on each side of the boundary is described by equation \ref{eq:phasedif} \cite{Feynman_Leighton_Sands_2013}.
\begin{equation}
    \dot\theta_2 - \dot\theta_1 = \frac{qV}{\hbar} = \dot\delta
    \label{eq:phasedif}
\end{equation}
Integrating on both sides yields equation \ref{eq:phaseVoltage}:
\begin{equation}
    \delta(t) = \delta_0 + \frac{q}{\hbar}\int V(t) dt
    \label{eq:phaseVoltage}
\end{equation}
From equation \ref{eq:phaseVoltage} one can conclude that for no applied voltage over the junction there is only a constant phase difference.
\subsubsection{SQUID's}
An application of the Josephson effect is the superconducting quantum interference device (SQUID). In essence a SQUID refers to a superconducting ring that contains one or more Josephson junction. The interference of superconducting wave functions across the Josephson junction allows for some useful applications discussed in detail below. 
\newline
\textit{Flux quantisation}
\newline
To understand the basic operation of a SQUID, one must first understand the concept of flux quantisation. To do so we consider a superconducting ring in the presence of a uniform magnetic field. The ring is superconducting so it exhibits the Meisner effect and thus the current density inside the ring is zero. Recall that the flux through a ring is:
\begin{equation}
    \Phi = \oint \Vec{A} \cdot d\Vec{s}
\end{equation}
Now consider the equation for the current density in a superconductor \cite{Feynman_Leighton_Sands_2013}: 
\begin{equation}
    \Vec{J} = \frac{\rho\hbar}{m}(\nabla\theta - \frac{q\Vec{A}}{\hbar})
    \label{eq:current}
\end{equation}
The current density inside the ring in the superconducting state is zero so equation \label{eq:current} becomes:
\begin{equation}
    \nabla\theta = \frac{q}{\hbar}\Vec{A}
    \label{eq:zeroCur}
\end{equation}
Integrating on both sides around a curve deep inside the superconductor such that the assumption that the current density is zero holds we can express equation \ref{eq:zeroCur} as:
\begin{equation}
    \oint\nabla\theta\cdot d\Vec{s} = \frac{q\Phi}{\hbar}
    \label{eq:intCurrent}
\end{equation}
Recognizing $\nabla\theta$ as vector field with potential function $\theta$, we can simply write the left hand side of equation \ref{eq:intCurrent} as $\theta(\Vec{r_1}) - \theta(\Vec{r_1})$. One might assume that the left hand side of the equation \ref{eq:intCurrent} must be equal to zero. This is incorrect because the absolute phase, meaning the potential function $\theta$ cannot be determined and can only be determined up to some constant because the gradient of the scalar potential function will take any constant to zero. Therefore the phase can only be determined relative to some point. According to \cite{Feynman_Leighton_Sands_2013} the only physical limitation we can place on the phase of the wave function is that the wave function must be singularly valued. This is intuitively understood as the fact that a particle cannot have two different amplitudes to be in a certain quantum state as that would imply that it has two probabilities associated with the exact same quantum state. Recalling equation \ref{eq:ensembleWave} one can conclude that the phase can change by any integer multiple of $2\pi$ as this results in the exact same wave function. Equation \ref{eq:intCurrent} then becomes:
\begin{equation}
    2\pi n = \frac{q\Phi}{\hbar}
    \label{eq:fluxQuant}
\end{equation}
Clearly equation \ref{eq:fluxQuant} implies that the flux through the superconducting loop must be quantized. 
\newline
\textit{The SQUID}
\newline
building
\newline
\textit{SQUID design considerations}
\newpage
\section{Literature Study}
\subsubsection{Noise in SQUIDs}
\subsubsection{InductEx}
\subsubsection{VTK file format}
\section{Implementation}


